\section{Methods}
The gene family trees represent orthogroups or root HOGs <ref>, that is, a tree describing all sequences assumed to
share a common ancestral sequence (including paralogs, or several individuals from the same population). These trees are
the input to the algorithm and may have been estimated by any phylogenetic method --- the algorithm is agnostic to the
source of disagreement (and therefore to the reason for the multiple leaves with same species label) or to the inference
procedure. The reference trees represent possible species trees, and must be on the same set of all species (i.e. all
species must be present in all reference trees). The set of reference trees should capture the variability between gene
trees with respect to the set of distances used to describe them. Therefore a good choice would be a set of optimal
species trees for the most dissimilar sets of gene families, although here we usually limit ourselves to a few species
trees inferred from a single set of gene families, with further randomisation in a few cases. It is important to notice
that the reference species trees are not restricted to the optimal ones (under some notion of optimality) and do not
need to include all optimal species trees. However having good candidates for the species trees will help interpreting
the gene trees in terms of them. The gene trees are assumed to be unrooted since the basic phylogenetic inference models
can’t infer the root location, but our method can be easily adapted for rooted gene family trees. The reference species
trees are rooted, although some distances disregard this information. The reference species trees are fixed beforehand,
but once they are set they can be used to create the “tree signal” of any gene family online, as long as all species
present in the gene family are represented in the reference species trees.

\red{describe the vector}
\begin{equation}
v(t_j,T) = \{d_i(t_j, T) \forall T \in T, i=\left(dups, loss, ils, RF, SPR\right) \} \forall j=1\dots k
\end{equation}


i.e. the spectral signature of gene tree $t_j$ is the set of combinations of distances i and reference trees T in a
specific order. In the simplest case, a single distance and a single tree could be used, as e.g. their RF distance to a
point estimate of their species tree. This would lead to a one-dimensional spectral signature, which may not be able to
discriminate well different gene trees. A natural extension is then to use more reference trees, in such a way that gene
trees now can be closer to one reference or another, or to use more distances, that will describe distinct ways in which
the gene trees relate to the reference.

\subsection{Distances}
We implemented several gene-species tree distances, in particular the reconciliation-based ones available in <guenomu>,
the multi-labelled tree version of the Robinson-Foulds distance called mulRF <citation>, and  two new distances based on
the the same ‘extended’ species tree from the mulRF distance, namely mulSPR and mulHdist. All distances utilise the
topological information only, neglecting branch lengths. Although we refer to them as ‘distances’ they are in fact
dissimilarity measures, and not true distance metrics since they do not satisfy the symmetry condition. The three
reconciliation-based distances implemented are the minimum number of duplications, minimum number of losses, and minimum
number of deep coalescences (also called ancestral polymorphisms or incomplete lineage sortings), and are based on the
LCA mapping between gene tree and species tree nodes. Since they originally work on rooted gene trees, we try all
possible gene rootings and store the minimal distance among all possibilities --- the species tree is rooted, however.

The other three distances are based on the concept of an ‘extended’ species tree, which replaces each leaf from the
species tree by a multifurcation with all leaves from the gene family tree mapping to this species (leaf) <citation>.
The resulting extended species tree and the gene family tree will have thus a one-to-one mapping between leaves. The
mulRF distance is then just the (unrooted) Robinson-Foulds distance between these trees. The mulSPR and the mulHdist,
equivalently, are just the SPR and Hdist distances between the gene family tree and the extended species tree: the SPR
distance is the approximation described in <citation>, and the Hdist is the total cost of matching each edge from the
gene tree to its optimal counterpart on the species tree (it has its name since we use the Hungarian method for solving
this assignment problem). This matching is also used by the SPR algorithm to find the smallest disagreement split
<citation>. The Hdist represents therefore the minimum number of leaves in disagreement per edge, summed over all edges.
If the gene tree is not multi-labelled (i.e. all leaves map to distinct leaves on the species tree) then the mulSPR and
the mulHdist correspond to the SPR and Hdist distances implemented in <citation>. These distances neglect the species
tree root location, treating it as unrooted.

\subsection{Normalisation}
Since we are comparing gene family trees with different numbers of leaves, species, and leaves per species, we must
rescale the distances to a common range. For all implemented distances, lower and upper bounds can be found, but in
practice they are too permissive and therefore we resort to randomisation to find tight bounds for the distances. This
is achieved by, for each gene family t and set of reference species trees T, after calculating the distances $d_i(t, T)$
for all distances i and species trees T, generate a new set of random species trees $\tau$ and calculate the distances to
these trees. We implemented two independent normalisations, that are concatenated into the vector of tree signals: the
p-value distance and the MinMax rescaled distance. The ‘p-value’ distance $p_i(t,T)$ counts the fraction of species trees
in total (i.e. amongst both randomised $\tau$ and reference T) presented a distance as small as the observed:

\begin{equation}
p_i(t,T) = \sum_{\tau and T} I (d_i(t,k) \leq d_i{t,T)) / |\tau and T|
\end{equation}

The MinMax distance $m_i(t,T)$ is the original distance rescaled to the zero-one interval:
\begin{equation}
m_i(t,T) = ( d_i(t,T) - Min ) / (Max - Min)
\end{equation}
where min and max are, respectively, the minimum and maximum distances observed among both $d_i(t,T)$ and $d_i(t,\tau)$.

The vector is thus
\begin{equation}
v(t_j,T) = \{\{p_i(t_j, T), m_i(t_j,T)\}  \forall T\in T, i=\left(dups, loss, ils, RF, SPR\right) \} forall j=1\dots k
\end{equation}

\subsection{Choice of reference trees}
Once the distances (dissimilarity measures in fact) and the reference trees are defined, we can calculate the signature
of each gene tree independently. The more distances are used the more we may discriminate between the gene trees, and
therefore we can implement and use as many measures as possible --- remembering that there are just a few that can be used
in general for a gene tree/species tree pair. However the choice of the reference trees may affect our ability to
distinguish between gene trees, since our rationale is that gene trees can be compared in terms of the species trees
they support according to each biological phenomenon. Given N species, the number of possible rooted reference trees is
r(N)=(2N-3)\!\!, which already exceeds $10^7$ even for N=10. However, in practice we don’t need that many reference trees
since many of them will not be supported by any gene tree in the sample. Ideally our set of reference trees is then all
species trees that can be supported by at least one gene tree in the sample. That is, if there is a distance measure i
and gene tree t  s.t. $d_i(t,T’) < \epsilon(i,t)$ then T’ should be added to the list of reference trees (where
$\epsilon(i,t)$ is a
small, arbitrary value in the distribution of distances $d_i$ over all possible species trees given t). Since finding such
trees T’ can challenging as well, a good strategy seems to be to find trees that summarise the information from subsets
of genes from the sample. One example is to use methods that incorporate uncertainty and therefore output sets of
species trees instead of finding a single point estimate (De Oliveira Martins, Mallo, and Posada 2016). Notice that even
random trees might be at distinct distances from two given gene trees, but this distinction depends also on the
discriminatory power of the distance measure. Also, we expect that by using collections of gene trees in the reference
tree search instead of each gene tree we reduce the number of possible species trees, since small, incomplete gene trees
are not informative in the sense that they favour equally a large number of species trees (not necessarily close to
other gene trees).

Notice that this relies on knowing the set of gene trees in advance, which somehow weakens our argument for continuous
update, but in practice we hope that a broad choice of reference trees (by using not only the ‘optimal’ ones) can
alleviate this problem. <needs rephrasing>

