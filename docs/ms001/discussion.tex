% Christophe comment: - acknowledge weaknesses of the approach (e.g. interpretation of mulSPR, not that many distances
% available (e.g. how to capture rooting errors?)
% - alternative to what we do is mechanistic approach french-swedish school DLTI+Species tree. complementary to this.
\section{Discussion}
Euclid’s first theorem states that “things which are equal to the same thing are equal to each other”.  We make use of
this principle to compare gene family trees, which cannot be compared to each other but that can be compared to the same
thing – species trees. They will be equal, however, only in terms of the biologically-inspired dissimilarity measures
that we can compute, and therefore we should use as many measures as possible.

We propose a method that allows for the partition/visualisation of gene families integrated with the species tree estimation.

As has been noted before (Gori et al. 2016), the SPR distance is a strong candidate for describing differences between
trees in the context of gene partitioning, and here we show that indeed it can be successfully used to visualise and
cluster gene families.

Our proposed algorithm is a combination of concatenation of multiple views (Zhao et al. 2017; Xu, Tao, and Xu 2013)
(where each gene-species tree pair dissimilarity is a view) and centroidQR or landmark-based representation (Chen and
Cai 2011; Rafailidis, Constantinou, and Manolopoulos 2017; Vin De Silva 2003), where the landmarks are sensible species
trees. As such, it can be improved through recent developments on both multi-view learning and landmark-based manifolds.
In special we expect more improvement over the choice of reference trees, and its relation to species tree inference. In
particular we are working on an iterative version of our algorithm where the gene tree space informs species tree
inference algorithms, which in turn may allow for more detailed tree spaces.

All results here employed unsupervised learning, but it is now straightforward to extend the analysis to supervised
cases, where the samples may belong to e.g. posterior distributions of gene trees and each class is a gene family.
Another idea is to split the gene family trees into well-supported and weakly-supported classes, to see if there is a
reasonable separation between them. Since the features are associated to species trees, feature selection techniques can
be interpreted in terms of the underlying species tree signals.


% Christophe's comment: expand this discussion: - choice of species tree indeed influence result, but noteworthy that
% the embedding and clustering was somewhat robust (fig. 2) ?
It is important to notice that we are not interested in the (unconditional, theoretical problem of) comparison between
trees, but in the specific problem of gene tree comparison conditional on possible scenarios of species trees. This
means that choosing the set of relevant species trees will influence the resulting space, and that it should be like
this. <feature, not a bug>

